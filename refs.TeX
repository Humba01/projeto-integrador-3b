%abnt refs
% KIBLER, Steven. HAUER, Andrew. GIESSEL, David. MALVEAUX, Chloe. RASKOVIC, Dejan. **IEEE Micromouse for mechatronics research and education**. *In:* **2011 IEEE International Conference on Mechatronics**. nº 11, 2011, Istambul, Turquia. IEEE Micromouse. Istambul: IEEE. 2011, p. 887-892.

% MISHRA, Swati. BANDE, Pankaj. **Maze Solving Algorithms for Micro Mouse**. *In:* **2008 IEEE International Conference on Signal Image Technology and Internet Based Systems**.  nº4, 2008, Bali, Indonesia. IEEE International Conference. Bali: IEEE. 2008, p. 86-93.

% referencias web
% "www.freecodecamp.org/portuguese/news/principios-de-programacao-orientada-a-objetos-em-java-conceitos-de-poo-para-iniciantes/"
% "www.blogson.com.br/classes-objetos-atributos-e-metodos-em-java/"
% "www.devmedia.com.br/explorando-a-classe-arraylist-no-java/24298"

% \bibliographystyle{sbc}
% \bibliography{sbc-template}
\bibitem{b1} S. Kibler, A. Hauer, D. Giessel, C. Malveaux, and D. Raskovic, ``IEEE Micromouse for mechatronics research and education,'' in 2011 IEEE International Conference on Mechatronics, 2011, pp. 887--892.
\bibitem{b2} S. Mishra and P. Bande, ``Maze Solving Algorithms for Micro Mouse,'' in 2008 IEEE International Conference on Signal Image Technology and Internet Based Systems, 2008, pp. 86--93.
\bibitem{b3} A. Oliveira, ``Princípios de Programação Orientada a Objetos em Java: Conceitos de POO para iniciantes,'' FreeCodeCamp, 2019. [Online]. Available: \url{https://www.freecodecamp.org/portuguese/news/principios-de-programacao-orientada-a-objetos-em-java-conceitos-de-poo-para-iniciantes/}. [Accessed: 18-Out-2023].
\bibitem{b4} T. MV, C. Leodegario, ``Classes, Objetos, Atributos e Métodos em Java,'' Blogson, 2019. [Online]. Available: \url{https://www.blogson.com.br/classes-objetos-atributos-e-metodos-em-java/}. [Accessed: 18-Out-2023].
\bibitem{b5} ``Explorando a classe ArrayList no Java,'' DevMedia, 2019. [Online]. Available: \url{https://www.devmedia.com.br/explorando-a-classe-arraylist-no-java/24298}. [Accessed: 22-Out-2023].
\bibitem{b6} M. Santos, "O Que é Mineração?", 2019. [Online]. Available: https://blog.jazida.com/o-que-e-mineracao/. [Accessed: 25- Out- 2023].
% \url{https://www.youtube.com/watch?v=YQs6IC-vgmo}. {Bjarne Stroustrup: Why you should avoid Linked Lists}
\bibitem{b7} B. Stroustrup, ``Why you should avoid Linked Lists,'' YouTube, 2019. [Online]. Available: \url{https://www.youtube.com/watch?v=YQs6IC-vgmo}. [Accessed: 25-Out-2023].



% https://www.bcc.unifal-mg.edu.br/~humberto/disciplinas/2010_2_grafos/pdf_aulas/aula_04.pdf
% https://tecnoblog.net/responde/o-que-e-uma-api-guia-para-iniciantes/
% https://www.alura.com.br/artigos/java
% https://rockcontent.com/br/blog/json/
% https://kenzie.com.br/blog/o-que-array/
\bibitem{b8} H. Sandmann, ``Teoria dos Grafos,'' Universidade Federal de Alfenas, 2010. [Online]. Available: \url{https://www.bcc.unifal-mg.edu.br/~humberto/disciplinas/2010_2_grafos/pdf_aulas/aula_04.pdf}. [Accessed: 18-Nov-2023].
\bibitem{b9} ``O que é uma API? Guia para iniciantes.'' Tecnoblog, 2019. [Online]. Available: \url{https://tecnoblog.net/responde/o-que-e-uma-api-guia-para-iniciantes/}. [Accessed: 18-Nov-2023].
\bibitem{b10} ``Java: o que é, para que serve e como funciona.'' Alura, 2019. [Online]. Available: \url{https://www.alura.com.br/artigos/java}. [Accessed: 18-Nov-2023].
\bibitem{b11} ``O que é JSON?'' Rock Content, 2019. [Online]. Available: \url{https://rockcontent.com/br/blog/json/}. [Accessed: 18-Nov-2023].
\bibitem{b12} ``O que é um Array?'' Kenzie Academy Brasil, 2019. [Online]. Available: \url{https://kenzie.com.br/blog/o-que-array/}. [Accessed: 18-Nov-2023].

% https://ufabcdivulgaciencia.proec.ufabc.edu.br/2021/06/04/prazer-me-chamo-teoria-dos-grafos-v-4-n-6-p-3-2021/
% \bibitem{b15} A. C. P. L. F. Carvalho, ``Prazer, me chamo Teoria dos Grafos,'' UFABC Divulga Ciência, 2021. [Online]. Disponível em: https://ufabcdivulgaciencia.proec.ufabc.edu.br/2021/06/04/prazer-me-chamo-teoria-dos-grafos-v-4-n-6-p-3-2021/. [Acesso em: 18-Nov-2023].





% https://repositorio.ufpb.br/jspui/bitstream/tede/7549/5/arquivototal.pdf
% https://tecnoblog.net/responde/o-que-e-algoritmo/
% https://www.prepbytes.com/blog/graphs/graph-in-data-structure/
% https://www.freecodecamp.org/portuguese/news/o-que-e-hashing-como-funcionam-os-codigos-de-hashing-com-exemplos/#:~:text=Em%20tabelas%20hash%20(em%20ingl%C3%AAs,o%20tamanho%20fixo%20que%20temos
% https://www.ime.usp.br/~pf/algoritmos/aulas/lista.html
\bibitem{b13} G. S. Melo, ``Introdução à Teoria dos Grafos,'' Universidade Federal da Paraíba, 2015. [Online]. Disponível em: https://repositorio.ufpb.br/jspui/bitstream/tede/7549/5/arquivototal.pdf. [Acesso em: 18-Nov-2023].
\bibitem{b14} ``O que é algoritmo?'' Tecnoblog, 2019. [Online]. Disponível em: https://tecnoblog.net/responde/o-que-e-algoritmo/. [Acesso em: 18-Nov-2023].
\bibitem{b15} ``Graph in Data Structure,'' PrepBytes, 2019. [Online]. Disponível em: https://www.prepbytes.com/blog/graphs/graph-in-data-structure/. [Acesso em: 18-Nov-2023].
\bibitem{b16} A. Oliveira, ``O que é Hashing? Como funcionam os códigos de Hashing com exemplos,'' FreeCodeCamp, 2019. [Online]. Disponível em: https://www.freecodecamp.org/portuguese/news/o-que-e-hashing-como-funcionam-os-codigos-de-hashing-com-exemplos/. [Acesso em: 18-Nov-2023].



\bibitem{b19} P. Feofiloff, ``Listas Encadeadas.'' IME-USP, 2018. [Online]. Disponível em: https://www.ime.usp.br/~pf/algoritmos/aulas/lista.html. [Acesso em: 18-Nov-2023].

% https://growthcode.com.br/algoritmos/pilha-estrutura-de-dados/
\bibitem{b20} W. Silva, ``Pilha: Estrutura de Dados,'' Growth Hackers, 2021. [Online]. Disponível em: https://growthcode.com.br/algoritmos/pilha-estrutura-de-dados/. [Acesso em: 18-Nov-2023].

% https://medium.com/luizalabs/java-garbage-collector-porque-precisamos-conhec%C3%AA-lo-9d26ebb0a6d8
\bibitem{b21} G. B. Zarelli, ``Java Garbage Collector — Por que precisamos conhecê-lo?'' Medium, 2022. [Online]. Disponível em: https://medium.com/luizalabs/java-garbage-collector-porque-precisamos-conhec\%C3\%AA-lo-9d26ebb0a6d8. [Acesso em: 18-Nov-2023].

% livro sistemas operacionais modernos Tanebaum
\bibitem{b22} A. S. Tanenbaum, ``Sistemas Operacionais Modernos,'' 4ª edição, 2015. pp. 134-135.
